\documentclass[]{article}
\usepackage{lmodern}
\usepackage{amssymb,amsmath}
\usepackage{ifxetex,ifluatex}
\usepackage{fixltx2e} % provides \textsubscript
\ifnum 0\ifxetex 1\fi\ifluatex 1\fi=0 % if pdftex
  \usepackage[T1]{fontenc}
  \usepackage[utf8]{inputenc}
\else % if luatex or xelatex
  \ifxetex
    \usepackage{mathspec}
  \else
    \usepackage{fontspec}
  \fi
  \defaultfontfeatures{Ligatures=TeX,Scale=MatchLowercase}
\fi
% use upquote if available, for straight quotes in verbatim environments
\IfFileExists{upquote.sty}{\usepackage{upquote}}{}
% use microtype if available
\IfFileExists{microtype.sty}{%
\usepackage{microtype}
\UseMicrotypeSet[protrusion]{basicmath} % disable protrusion for tt fonts
}{}
\usepackage[margin=1in]{geometry}
\usepackage{hyperref}
\hypersetup{unicode=true,
            pdftitle={R Notebook},
            pdfborder={0 0 0},
            breaklinks=true}
\urlstyle{same}  % don't use monospace font for urls
\usepackage{color}
\usepackage{fancyvrb}
\newcommand{\VerbBar}{|}
\newcommand{\VERB}{\Verb[commandchars=\\\{\}]}
\DefineVerbatimEnvironment{Highlighting}{Verbatim}{commandchars=\\\{\}}
% Add ',fontsize=\small' for more characters per line
\usepackage{framed}
\definecolor{shadecolor}{RGB}{248,248,248}
\newenvironment{Shaded}{\begin{snugshade}}{\end{snugshade}}
\newcommand{\KeywordTok}[1]{\textcolor[rgb]{0.13,0.29,0.53}{\textbf{#1}}}
\newcommand{\DataTypeTok}[1]{\textcolor[rgb]{0.13,0.29,0.53}{#1}}
\newcommand{\DecValTok}[1]{\textcolor[rgb]{0.00,0.00,0.81}{#1}}
\newcommand{\BaseNTok}[1]{\textcolor[rgb]{0.00,0.00,0.81}{#1}}
\newcommand{\FloatTok}[1]{\textcolor[rgb]{0.00,0.00,0.81}{#1}}
\newcommand{\ConstantTok}[1]{\textcolor[rgb]{0.00,0.00,0.00}{#1}}
\newcommand{\CharTok}[1]{\textcolor[rgb]{0.31,0.60,0.02}{#1}}
\newcommand{\SpecialCharTok}[1]{\textcolor[rgb]{0.00,0.00,0.00}{#1}}
\newcommand{\StringTok}[1]{\textcolor[rgb]{0.31,0.60,0.02}{#1}}
\newcommand{\VerbatimStringTok}[1]{\textcolor[rgb]{0.31,0.60,0.02}{#1}}
\newcommand{\SpecialStringTok}[1]{\textcolor[rgb]{0.31,0.60,0.02}{#1}}
\newcommand{\ImportTok}[1]{#1}
\newcommand{\CommentTok}[1]{\textcolor[rgb]{0.56,0.35,0.01}{\textit{#1}}}
\newcommand{\DocumentationTok}[1]{\textcolor[rgb]{0.56,0.35,0.01}{\textbf{\textit{#1}}}}
\newcommand{\AnnotationTok}[1]{\textcolor[rgb]{0.56,0.35,0.01}{\textbf{\textit{#1}}}}
\newcommand{\CommentVarTok}[1]{\textcolor[rgb]{0.56,0.35,0.01}{\textbf{\textit{#1}}}}
\newcommand{\OtherTok}[1]{\textcolor[rgb]{0.56,0.35,0.01}{#1}}
\newcommand{\FunctionTok}[1]{\textcolor[rgb]{0.00,0.00,0.00}{#1}}
\newcommand{\VariableTok}[1]{\textcolor[rgb]{0.00,0.00,0.00}{#1}}
\newcommand{\ControlFlowTok}[1]{\textcolor[rgb]{0.13,0.29,0.53}{\textbf{#1}}}
\newcommand{\OperatorTok}[1]{\textcolor[rgb]{0.81,0.36,0.00}{\textbf{#1}}}
\newcommand{\BuiltInTok}[1]{#1}
\newcommand{\ExtensionTok}[1]{#1}
\newcommand{\PreprocessorTok}[1]{\textcolor[rgb]{0.56,0.35,0.01}{\textit{#1}}}
\newcommand{\AttributeTok}[1]{\textcolor[rgb]{0.77,0.63,0.00}{#1}}
\newcommand{\RegionMarkerTok}[1]{#1}
\newcommand{\InformationTok}[1]{\textcolor[rgb]{0.56,0.35,0.01}{\textbf{\textit{#1}}}}
\newcommand{\WarningTok}[1]{\textcolor[rgb]{0.56,0.35,0.01}{\textbf{\textit{#1}}}}
\newcommand{\AlertTok}[1]{\textcolor[rgb]{0.94,0.16,0.16}{#1}}
\newcommand{\ErrorTok}[1]{\textcolor[rgb]{0.64,0.00,0.00}{\textbf{#1}}}
\newcommand{\NormalTok}[1]{#1}
\usepackage{graphicx,grffile}
\makeatletter
\def\maxwidth{\ifdim\Gin@nat@width>\linewidth\linewidth\else\Gin@nat@width\fi}
\def\maxheight{\ifdim\Gin@nat@height>\textheight\textheight\else\Gin@nat@height\fi}
\makeatother
% Scale images if necessary, so that they will not overflow the page
% margins by default, and it is still possible to overwrite the defaults
% using explicit options in \includegraphics[width, height, ...]{}
\setkeys{Gin}{width=\maxwidth,height=\maxheight,keepaspectratio}
\IfFileExists{parskip.sty}{%
\usepackage{parskip}
}{% else
\setlength{\parindent}{0pt}
\setlength{\parskip}{6pt plus 2pt minus 1pt}
}
\setlength{\emergencystretch}{3em}  % prevent overfull lines
\providecommand{\tightlist}{%
  \setlength{\itemsep}{0pt}\setlength{\parskip}{0pt}}
\setcounter{secnumdepth}{0}
% Redefines (sub)paragraphs to behave more like sections
\ifx\paragraph\undefined\else
\let\oldparagraph\paragraph
\renewcommand{\paragraph}[1]{\oldparagraph{#1}\mbox{}}
\fi
\ifx\subparagraph\undefined\else
\let\oldsubparagraph\subparagraph
\renewcommand{\subparagraph}[1]{\oldsubparagraph{#1}\mbox{}}
\fi

%%% Use protect on footnotes to avoid problems with footnotes in titles
\let\rmarkdownfootnote\footnote%
\def\footnote{\protect\rmarkdownfootnote}

%%% Change title format to be more compact
\usepackage{titling}

% Create subtitle command for use in maketitle
\newcommand{\subtitle}[1]{
  \posttitle{
    \begin{center}\large#1\end{center}
    }
}

\setlength{\droptitle}{-2em}
  \title{R Notebook}
  \pretitle{\vspace{\droptitle}\centering\huge}
  \posttitle{\par}
  \author{}
  \preauthor{}\postauthor{}
  \date{}
  \predate{}\postdate{}


\begin{document}
\maketitle

This is an \href{http://rmarkdown.rstudio.com}{R Markdown} Notebook.
When you execute code within the notebook, the results appear beneath
the code.

Try executing this chunk by clicking the \emph{Run} button within the
chunk or by placing your cursor inside it and pressing
\emph{Ctrl+Shift+Enter}.

\begin{Shaded}
\begin{Highlighting}[]
\KeywordTok{plot}\NormalTok{(cars)}
\end{Highlighting}
\end{Shaded}

\includegraphics{10_files/figure-latex/unnamed-chunk-1-1.pdf}

Add a new chunk by clicking the \emph{Insert Chunk} button on the
toolbar or by pressing \emph{Ctrl+Alt+I}.

When you save the notebook, an HTML file containing the code and output
will be saved alongside it (click the \emph{Preview} button or press
\emph{Ctrl+Shift+K} to preview the HTML file).

The preview shows you a rendered HTML copy of the contents of the
editor. Consequently, unlike \emph{Knit}, \emph{Preview} does not run
any R code chunks. Instead, the output of the chunk when it was last run
in the editor is displayed.

\begin{Shaded}
\begin{Highlighting}[]
\KeywordTok{options}\NormalTok{(}\DataTypeTok{warn =} \OperatorTok{-}\DecValTok{1}\NormalTok{)}
\KeywordTok{library}\NormalTok{(ISLR)}
\KeywordTok{summary}\NormalTok{(Weekly)}
\end{Highlighting}
\end{Shaded}

\begin{verbatim}
##       Year           Lag1               Lag2               Lag3         
##  Min.   :1990   Min.   :-18.1950   Min.   :-18.1950   Min.   :-18.1950  
##  1st Qu.:1995   1st Qu.: -1.1540   1st Qu.: -1.1540   1st Qu.: -1.1580  
##  Median :2000   Median :  0.2410   Median :  0.2410   Median :  0.2410  
##  Mean   :2000   Mean   :  0.1506   Mean   :  0.1511   Mean   :  0.1472  
##  3rd Qu.:2005   3rd Qu.:  1.4050   3rd Qu.:  1.4090   3rd Qu.:  1.4090  
##  Max.   :2010   Max.   : 12.0260   Max.   : 12.0260   Max.   : 12.0260  
##       Lag4               Lag5              Volume       
##  Min.   :-18.1950   Min.   :-18.1950   Min.   :0.08747  
##  1st Qu.: -1.1580   1st Qu.: -1.1660   1st Qu.:0.33202  
##  Median :  0.2380   Median :  0.2340   Median :1.00268  
##  Mean   :  0.1458   Mean   :  0.1399   Mean   :1.57462  
##  3rd Qu.:  1.4090   3rd Qu.:  1.4050   3rd Qu.:2.05373  
##  Max.   : 12.0260   Max.   : 12.0260   Max.   :9.32821  
##      Today          Direction 
##  Min.   :-18.1950   Down:484  
##  1st Qu.: -1.1540   Up  :605  
##  Median :  0.2410             
##  Mean   :  0.1499             
##  3rd Qu.:  1.4050             
##  Max.   : 12.0260
\end{verbatim}

\begin{Shaded}
\begin{Highlighting}[]
\KeywordTok{plot}\NormalTok{(Weekly)}
\end{Highlighting}
\end{Shaded}

\includegraphics{10_files/figure-latex/unnamed-chunk-3-1.pdf} checking
corealationship between variables. Excluding \texttt{Direction} which is
qualitative. It appears only \texttt{Year} and \texttt{Volume} have
relationship.

\begin{Shaded}
\begin{Highlighting}[]
\KeywordTok{cor}\NormalTok{(Weekly[, }\OperatorTok{-}\DecValTok{9}\NormalTok{])}
\end{Highlighting}
\end{Shaded}

\begin{verbatim}
##               Year         Lag1        Lag2        Lag3         Lag4
## Year    1.00000000 -0.032289274 -0.03339001 -0.03000649 -0.031127923
## Lag1   -0.03228927  1.000000000 -0.07485305  0.05863568 -0.071273876
## Lag2   -0.03339001 -0.074853051  1.00000000 -0.07572091  0.058381535
## Lag3   -0.03000649  0.058635682 -0.07572091  1.00000000 -0.075395865
## Lag4   -0.03112792 -0.071273876  0.05838153 -0.07539587  1.000000000
## Lag5   -0.03051910 -0.008183096 -0.07249948  0.06065717 -0.075675027
## Volume  0.84194162 -0.064951313 -0.08551314 -0.06928771 -0.061074617
## Today  -0.03245989 -0.075031842  0.05916672 -0.07124364 -0.007825873
##                Lag5      Volume        Today
## Year   -0.030519101  0.84194162 -0.032459894
## Lag1   -0.008183096 -0.06495131 -0.075031842
## Lag2   -0.072499482 -0.08551314  0.059166717
## Lag3    0.060657175 -0.06928771 -0.071243639
## Lag4   -0.075675027 -0.06107462 -0.007825873
## Lag5    1.000000000 -0.05851741  0.011012698
## Volume -0.058517414  1.00000000 -0.033077783
## Today   0.011012698 -0.03307778  1.000000000
\end{verbatim}

\begin{Shaded}
\begin{Highlighting}[]
\KeywordTok{attach}\NormalTok{(Weekly)}
\NormalTok{glm_fit =}\StringTok{ }\KeywordTok{glm}\NormalTok{(Direction }\OperatorTok{~}\StringTok{ }\NormalTok{Lag1 }\OperatorTok{+}\StringTok{ }\NormalTok{Lag2 }\OperatorTok{+}\StringTok{ }\NormalTok{Lag3 }\OperatorTok{+}\StringTok{ }\NormalTok{Lag4 }\OperatorTok{+}\StringTok{ }\NormalTok{Lag5 }\OperatorTok{+}\StringTok{ }\NormalTok{Volume, }\DataTypeTok{data =}\NormalTok{ Weekly, }\DataTypeTok{family =}\NormalTok{ binomial)}
\KeywordTok{summary}\NormalTok{(glm_fit)}
\end{Highlighting}
\end{Shaded}

\begin{verbatim}
## 
## Call:
## glm(formula = Direction ~ Lag1 + Lag2 + Lag3 + Lag4 + Lag5 + 
##     Volume, family = binomial, data = Weekly)
## 
## Deviance Residuals: 
##     Min       1Q   Median       3Q      Max  
## -1.6949  -1.2565   0.9913   1.0849   1.4579  
## 
## Coefficients:
##             Estimate Std. Error z value Pr(>|z|)   
## (Intercept)  0.26686    0.08593   3.106   0.0019 **
## Lag1        -0.04127    0.02641  -1.563   0.1181   
## Lag2         0.05844    0.02686   2.175   0.0296 * 
## Lag3        -0.01606    0.02666  -0.602   0.5469   
## Lag4        -0.02779    0.02646  -1.050   0.2937   
## Lag5        -0.01447    0.02638  -0.549   0.5833   
## Volume      -0.02274    0.03690  -0.616   0.5377   
## ---
## Signif. codes:  0 '***' 0.001 '**' 0.01 '*' 0.05 '.' 0.1 ' ' 1
## 
## (Dispersion parameter for binomial family taken to be 1)
## 
##     Null deviance: 1496.2  on 1088  degrees of freedom
## Residual deviance: 1486.4  on 1082  degrees of freedom
## AIC: 1500.4
## 
## Number of Fisher Scoring iterations: 4
\end{verbatim}

It appears only \texttt{Lag2} have some statistical significance.


\end{document}
